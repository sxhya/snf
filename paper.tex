\documentclass[oneside, 10pt]{amsart}
\usepackage{amscd, amsmath, amssymb, amsthm, amsfonts, amstext, geometry, verbatim, enumitem, graphicx, mathtools, xfrac, microtype, nameref, thmtools}
\usepackage[breaklinks=true]{hyperref}
\usepackage[capitalize]{cleveref}
\usepackage[hyperref=true, backend=bibtex, firstinits=true, citestyle=numeric-comp, sortlocale=en_US, url=false, doi=false, eprint=true, maxbibnames=4]{biblatex}            
\usepackage[matrix,arrow,curve]{xy}
\usepackage[toc,page]{appendix}

\addbibresource{paper.bib}
\renewbibmacro*{volume+number+eid}{\ifentrytype{article}{\- \iffieldundef{volume}{}{Vol.~\printfield{volume},}\iffieldundef{number}{}{ No.~\printfield{number},}}}
\renewbibmacro{in:}{\ifentrytype{article}{}{\printtext{\bibstring{in}\intitlepunct}}}
\newbibmacro{string+doi}[1]{\iffieldundef{doi}{\iffieldundef{url}{#1}{\href{\thefield{url}}{#1}}}{\href{http://dx.doi.org/\thefield{doi}}{#1}}}
\DeclareFieldFormat[article, inproceedings, inbook, book, thesis]{title}{\usebibmacro{string+doi}{\mkbibquote{#1}}}
\renewcommand*{\bibfont}{\footnotesize}
\renewcommand\thesubsection{\arabic{subsection}}

\theoremstyle{plain}

\newtheorem{thm}{Theorem}
\newtheorem{prop}{Proposition}
\Crefname{thm}{Theorem}{Theorems}

\newtheorem{lemma}{Lemma}
\Crefname{lemma}{Lemma}{Lemmas}

\newtheorem{cor}[lemma]{Corollary}
\Crefname{cor}{Corollary}{Corollaries}

\Crefname{prop}{Proposition}{Propositions}

\newtheorem*{thm*}{Theorem}
\newtheorem*{lemma*}{Lemma}

\theoremstyle{remark}

\newtheorem{rem}[lemma]{Remark}
\Crefname{rem}{Remark}{Remarks}

\theoremstyle{definition}

\newtheorem{df}[lemma]{Definition} \Crefname{df}{Definition}{Definitions}
\newtheorem{example}{Example} \Crefname{example}{Example}{Examples} 
 
\newcommand\SmallMatrix[1]{{\tiny\arraycolsep=0.3\arraycolsep\ensuremath{\begin{pmatrix}#1\end{pmatrix}}}}
\DeclareMathOperator{\St}{St}
\DeclareMathOperator{\SL}{SL}
\newcommand{\ZZ}{\mathbb{Z}}
\newcommand{\K}{K_2}
\newcommand{\Kmw}{K^{\mathrm{MW}}_2}

\title{A note on $\K$ of Kac---Moody groups over a field}
\author {Sergey Sinchuk}
\keywords {$K_2$-functor, Kac---Moody groups, Smith normal form,
 {\em Mathematical Subject Classification (2010):} 15A21, 20G44, 19C20}
\address{Chebyshev laboratory, St. Petersburg State University, St. Petersburg, Russia}
\email {sinchukss at gmail.com}
\thanks {This work received financial support from the Russian Science Foundation grant 14-21-00035.}
\date {\today}

\begin{document}   
\begin{abstract} We describe a method of computing the $\K$-group associated to any Kac--Moody group over a field.
 The key ingredient in our computation is a variant of Smith normal form of an integer matrix.
\end{abstract}


\maketitle
The aim of this note is to compute the unstable $\K$-group $\K(A, F)$ for arbitrary generalized Cartan matrix $A$ and $F$ a field,
 and thus give a complete answer to the question raised in~\cite{MW}.
Recall from~\cite{MR}, \cite{Ti} that, by definition, the group $\K(A, F)$ is the kernel of the map 
 $\St(A, F) \to G(A, F)$ between the corresponding Steinberg and Kac--Moody groups over $F$.

In~\cite{MR} J.~Morita and U.~Rehmann described $\K(A, F)$ as a certain quotient of abelian groups, see~\cref{thm:mr} below.
Using this description, the group $\K(A, F)$ can be easily computed for finite or affine $A$ (see Corollaries~4--5 of \cite{MR}).
For general $A$, however, this description is only indirect and one is required to do some computation in order to get an explicit answer.
In~\cite{MW} M.~Westaway performed such computation for all {\it hyperbolic} GCMs, a subclass of GCMs of indefinite type. 
Westaway's proof involved case-by-case analysis and, moreover, 
 relied on the classification of hyperbolic GCMs previously obtained by L.~Carbone et al.

In this note we describe a solution of the general problem, namely we describe a method to compute $\K(A, F)$ from the given GCM $A$.
We also reprove some of the results of~\cite{MW} in a simplified form.
Our method is, essentially, a variant of the well-known algorithm for computing Smith normal form of a matrix.

\subsection{Preliminaries}
Throughout this note $F$ denotes an arbitrary field and $A=(a_{ij})$ a generalized Cartan matrix of size $n$.
Denote by $M_A$ the subgroup of the free abelian group $\ZZ^n = \ZZ e_1 \oplus \ldots \oplus \ZZ e_n $ generated by all 
 linear combinations of the form $a_{ji} e_i - a_{ij} e_j$, $1\leq i<j\leq n$.
Set $G_A = \ZZ^n/M_A$.

Let us first recall the precise statement of the result of Morita and Rehmann.
Denote by $L_i$ either the Milnor group $\K(F)$ or the Milnor--Witt group $\Kmw(F)\cong K_2(2, F)$ 
 depending on whether or not there exists $1 \leq k \leq n$ such that the matrix coefficient $a_{ki}$ is odd.
In either case we denote the Steinberg symbols (resp. cocycles) generating the group $L_i$ by $\{x,y\}_i$, $x, y\in F^\times$
 (see~\cite[Proposition~5.5, Lemme~5.6]{Ma} for the definition of Steinberg cocycles and symbols, respectively).
We always use additive notation for abelian groups $\K(F)$ or $\Kmw(F)$.
 
\begin{thm}[Morita--Rehmann] \label{thm:mr} 
 $\K(A, F)$ is isomorphic to the quotient $L/J$ where $L = L_1 \oplus L_2 \oplus \ldots \oplus L_n$ and 
 $J$ is the subgroup of $L$ generated by relators $\{x, y^{a_{ji}}\}_i - \{x^{a_{ij}}, y\}_j = 0$.
\end{thm}

Now, let us recall how $\K(A, F)$ can be computed in the special case when each column of $A$ contains an odd entry.
The following result is proved in~\cite{MW} (cf.~Theorem~5.1]).
\begin{thm} \label{thm-odd}
  Assume that every column of $A$ contains an odd entry. Assume, moreover, that
  $G_A \cong \mathbb{Z}/r_1\mathbb{Z} \oplus \ldots \oplus \mathbb{Z}/r_s\mathbb{Z} \oplus \mathbb{Z}^{n-s}\text{ for some } r_i | r_{i+1}.$
  Then 
  \begin{equation} \label{thm-eq}
    \K(A, F) \cong \frac{\K(F)}{r_1 \K(F)} \oplus \ldots \oplus \frac{\K(F)}{r_s \K(F)} \oplus \K(F)^{n-s}.
  \end{equation}
\end{thm}
\begin{proof}
 Notice that $\{x, y^k\}_i = \{x^k, y\}_i = k\{x,y\}_i$ for all $k$ and $1\leq i\leq n$.
 
 We already know by~\cref{thm:mr} that $\K(A, F)$ is the quotient of the abelian group 
  $\K(F)^n \cong \K(F) \otimes_{\mathbb{Z}} (\mathbb{Z}e_1 \oplus \ldots \oplus \mathbb{Z}e_n)$ by the subgroup
  \[\langle a_{ji}\{u, v\}_i - a_{ij}\{u, v\}_j \rangle \cong \langle \{u, v\} \otimes (a_{ji} e_i - a_{ij} e_j)\rangle = \K(F) \otimes M_A. \]
 Since $\{u, v\}$ generate $\K(F)$ we get $\K(A, F) \cong \K(F) \otimes (\mathbb{Z}^n / M_A) = \K(F) \otimes G_A$,
  the latter expression coincides with the expression in the right-hand side of~\eqref{thm-eq}. 
\end{proof}

Of course, the integers $r_i$ can be concretely obtained as
  the diagonal coefficients in the Smith normal form of the $\frac{n(n-1)}{2}\times n$ matrix $\widetilde{A}$ given by
\begin{equation} \label{eq:Aijk} \widetilde{A}_{i<j, k} = \left\{\def\arraystretch{1.2}%
  \begin{array}{@{}c@{\quad}l@{}}
     a_{ij} & \text{ for } k = i, \\
    -a_{ji} & \text{ for } k = j, \\
    0 & \text{otherwise.}    
  \end{array}\right.\end{equation}                    
Notice that the integer $s$ from the statement of~\cref{thm-odd} is precisely the number of nonzero $r_i$'s.
  
In the case when all entries of $A$ are even, invariant factors $r_i$ are even as well and the group $\K(A, F)$ can be computed just as easily.
\begin{thm} \label{thm-even}
  Assume that all entries of $A$ are even.
  Assume, moreover, that $G_A \cong \ZZ / r_1 \ZZ \oplus \ldots \oplus \ZZ / r_s\ZZ \oplus \ZZ^{m-s}$ for some $r_i$ satisfying $r_i | r_{i+1}$.
  Then
   \[\K(A, F) \cong \frac{\Kmw(F)}{(r_1/2) I} \oplus \ldots \oplus \frac{\Kmw(F)}{(r_s/2) I} \oplus \Kmw(F)^{m-s}, \]
    where $I$ denotes the subgroup of $\Kmw(F)$ generated by $\{x^2, y\}$, $x, y\in F^\times$.
\end{thm}
\begin{proof}
 Recall from ~\cite[Lemma~3.2]{MW} that
 \[\{x, y^{2k}\}_i = \{x^{2k}, y\}_i = k\{x,y^2\}_i = k\{x^2, y\}_i \text{ for all } k\text{ and } 1\leq i\leq n.\]
 By~\cref{thm:mr} the group $\K(A, F)$ is the quotient of
  $\Kmw(F) \otimes \ZZ^n$ by the subgroup 
  \[\left\langle \frac{a_{ji}}{2}\{x^2, y\}_i - \frac{a_{ij}}{2}\{x^2, y\}_j \right\rangle \cong \left\langle 
   \{x^2, y\} \otimes \left(\frac{a_{ji}}{2} e_i - \frac{a_{ij}}{2} e_j\right)\right\rangle,\ x,y\in F^\times. \]
 Thus we get that $\K(A, F) \cong \left(\Kmw(F) \otimes \ZZ^n\right) / (I \otimes M_A/2),$ as claimed. 
\end{proof}
Notice that \cite[Theorem~6.1]{MW} formally follows from the above theorem.

\subsection{The main result}
Now we turn to the case when $A$ contains columns of both types 
(i.\,e. there are columns having an odd entry, and consisting only of even entries).
The basic idea remains the same: we need to find a nice basis for the subgroup $J\leq L$ of defining relations of $\K(A, F)$.
This time, however, the situation is more tricky since we cannot simply tensor out $\K(F)$ or $\Kmw(F)$.
To solve the problem we need to invent some analogue of the fundamental theorem of finitely-generated abelian groups suitable to our sitation.

We need to introduce the notion of a {\it matrix with parity}.
By definition, it is simply a pair $(b, p_b)$ consisting of an integer $m\times n$ matrix $b$ and a column function $p_b\colon \{1,\ldots,n\}\to \{odd, even\}$,
 such that entries $b_{ij}$ are even for all $i$ given $p_b(j)=even$.
We call function $p_b$ a {\it parity function}.
We also call the corresponding column $b_{*i}$ {\it odd} or {\it even} depending on the value of $p_b(i)$.
Notice that we make no assumptions about entries of odd columns.

We now list five types of elementary operations that can be applied to a $m \times n$ matrix with parity $(b, p_b)$.
Later we will see that it is possible to bring a matrix with parity to Smith normal form using these operations.
\begin{enumerate}
 \item \label{item:type1} Arbitrary row operation, i.\,e. the operation that changes $b$ into $gb$ for some $g \in \SL(m, \ZZ)$;
 \item \label{item:type2} Interchanging $i$-th and $j$-th columns of $b$ 
  (the corresponding values $p_b(i)$ and $p_b(j)$ of the parity function are also interchanged);
 \item \label{item:type3} Adding a multiple of a column to another column of the same parity;
 \item \label{item:type4} Adding a multiple of an even column to an odd column;
 \item \label{item:type5} Adding an even multiple of an odd column to an even column.
\end{enumerate}
All operations except the one of type~\eqref{item:type2} leave the parity function $p_b$ unchanged.
Operations \eqref{item:type3}--\eqref{item:type5} change $b$ into $bt_{ij}(\xi)$
 for some elementary transvection matrix $t_{ij}(k) =e+k e_{ij}$, $1\leq i\neq j\leq n$, $k \in \ZZ$.
Notice that the matrix produced by any of the above transformations is still a matrix with parity.

It is clear how matrix $\widetilde{A}$ can be reinterpreted as a matrix with parity.
Indeed, set $p_{\widetilde{A}}(i) = odd$ (resp. $p_{\widetilde{A}}(i) = even$) if the column $a_{*i}$ of
 $A$ contains an odd entry (resp. consists only of even entries).

Now let $(c, p_c)$ be an $m\times n$ matrix with parity.
Define abelian group $L(p_c)$ as the direct sum $L_1 \oplus \ldots \oplus L_n$, where 
\[L_i=\left\{\def\arraystretch{1.2}%
  \begin{array}{@{}c@{\quad}l@{}}
     \K(F) & \text{if $p_c(i) = odd$,} \\    
     \Kmw(F) & \text{if $p_c(i) = even$}.
  \end{array}\right.\]
As before, we use notation $\{u, v\}_i$ for the Steinberg symbol (or cocycle) of $i$-th summand $L_i$.
Define $J(c)$ to be the subgroup of $L(p_c)$ generated by the following elements:
\[ \{ u^{c_{i1}}, v\}_1 + \ldots + \{u^{c_{in}}, v\}_{n},\ \ u, v\in F^\times, 1\leq i \leq m.\]
It is clear that $L(p_{\widetilde{A}})$ and $J({\widetilde{A}})$ coincide with $L$ and $J$ from the statement of~\cref{thm:mr}.

\begin{lemma} \label{lm:inv} The isomorphism class of the quotient $L(p_c)/J(c)$ is not changed
 by any of the above five elementary transformations of $(c, p_c)$. \end{lemma}
\begin{proof}
 Let us verify the assertion of the lemma, for example, for the operation of type~\eqref{item:type5}
  (for other operations it is similar but easier).
 Assume that $c'$ is the matrix obtained from $c$ by applying some operation of type~\eqref{item:type5}.
 
 It suffices to show that $J(c')$ is the image of $J(c)$ under an automorphism of $L(p_c)$.
 Speaking in terms of bases, it suffices to describe how the ``new'' symbols $\{u, v\}_i'$ corresponding to $c'$ 
  are expressed in terms of ``old'' symbols $\{u, v\}_i$ corresponding to $c$.  
 
 By assumption, $c' = c \cdot t_{ij}(2k)$ for some $i$, $j$ satisfying $p_c(i)=odd$, $p_c(j)=even$.
 This operation corresponds to the following change of basis of $L(p_c)$:
 \[ \{u, v\}_{i}^{'} = \{u, v\}_i - 2k\{u, v\}_j;\ \ \{u, v\}_{j}^{'} = \{u, v\}_j.\]
 It remains to see that $\{u, v\}_{i}^{'}$ is a Steinberg symbol.
 Indeed, although $\{u, v\}_j$ is merely a cocycle, $2k\{u, v\}_j = \{u^{2k}, v\}_j$ is a symbol (cf.~\cite[Lemma~3.2(vii)]{MW}).
 Since $\{u, v\}_{i}^{'}$ is a linear combination of symbols, it is a symbol as well.
\end{proof}

\begin{lemma} \label{lm:normal-form} Any $m \times n$ integer matrix with parity $(c, p_c)$ can be transformed into a diagonal matrix $(d, p_d)$
  ($d = \mathrm{diag}(r_1, \ldots r_{\mathrm{min}(m, n)})$, $r_i | r_{i+1}$)
   by means of elementary operations of types~\eqref{item:type1}--\eqref{item:type5}.
\end{lemma}
\begin{proof} 
Recall that the classical algorithm for finding Smith normal form of a matrix involves
 operation of addition of one of the two fixed columns (say, $i$-th and $j$-th ones) to another
 so that, unless it was zero from the start, the minimum of $|c_{ki}|$ and $|c_{kj}|$ decreases strictly. 
 Here $k$ is some row number and $i$, $j$ are some column numbers selected by the algorithm on one if its steps.

To adapt this algorithm to our situation it suffices to describe
 how the above operation is performed in the case when $i$-th and $j$-th columns have different parities.
Indeed, the only difference between our elementary operations and the usual ones is that
 we do not allow adding odd multiples of odd columns to even columns.

Suppose that $i$-th column of $c$ is odd and $j$-th one is even.

We only need to consider the case $|c_{ki}| < |c_{kj}|$. Write $c_{kj} + q c_{ki} = r$ for some $q$ and $r$, satisfying $|r| < |c_{ki}|$.
If $q$ is even, add $q \cdot c_{*i}$ to $c_{*j}$ using an operation of type~\eqref{item:type5} and 
 notice that $\mathrm{min}(|c_{ki}|,|r|) < \mathrm{min}(|c_{ki}|,|c_{kj}|)$.
Otherwise, choose even $q'$ equal to either $q-1$ or $q+1$ so that one of the following is true:
\begin{itemize}
 \item $r\neq 0$ and $|c_{kj} + q' c_{ki}| < |c_{ki}|$;
 \item $r = 0$ and $c_{kj} + q' c_{ki} = c_{ki}$.
\end{itemize}
In the first case after adding $q' \cdot c_{*i}$ to $c_{*j}$ we obtain the required inequality.
In the second case we need to further subtract $c_{*j}$ from $c_{*i}$ to get $c_{ki} = 0$.
\end{proof}

Combining the above two lemmas we obtain the following result.
\begin{thm}
If $(\widetilde{A}, p_{\widetilde{A}})$ is reduced to Smith canonical form $(d, p_d)$, for some $d = \mathrm{diag}(r_1, \ldots r_n)$ then
 each invariant factor $r_i$ produces one summand in $\K(A, F)$ which, depending on $p_d(i)$, 
 is isomorphic to $\K(F)/r_i \K(F)$ or $\Kmw(F) / (r_i/2) I$.
\end{thm}

Notice that the total numbers of odd and even columns of $d$ are the same as the corresponding numbers for $A$ and $\widetilde{A}$.
In many cases this allows one to immediately compute $\K(A, F)$ only knowing SNF of $\widetilde{A}$.
\begin{example}
Consider hyperbolic GCM \textnumero~156 from~\cite[p.~19]{MW}.
 Set $A = \left(\begin{smallmatrix}
 2& 0& 0& -1\\
 0& 2& 0& -1\\
 0& 0& 2& -1\\
 -2& -2& -2& 2\end{smallmatrix}\right)$.
Clearly, $A$ has three even columns and one odd column. 
The Smith normal form of $\widetilde{A}$ is $d=\mathrm{diag}(1,2,2,0)$.
Notice that a diagonal coefficient of $d$ can be odd only if the corresponding column is odd.
Thus, we get that $p_d(1)=odd$, $p_d(i)=even$ for $i>1$, and we compute $\K(A, F)$ as $\Kmw(F)/I \oplus \Kmw(F)/I \oplus \Kmw(F)$,
 which agrees with~\cite{MW}.
\end{example}
  
Now suppose that columns of $A$ (and hence of $\widetilde{A}$) are ordered in such a way that
 each of the first $k$ columns contains and odd entry, and the remaining $n-k$ columns consist of even entries.
The algorithm given in~\cref{lm:normal-form} delivers SNF decomposition
 $\widetilde{A} = p \cdot d \cdot q $ for some $q \in \SL(n, \ZZ)$, $p \in \SL(n(n-1)/2, \ZZ)$, $d = \mathrm{diag}(r_1, \ldots r_n)$.

Looking at the effect of operations \eqref{item:type2}--\eqref{item:type5} modulo $2$ one can notice the following two properties of the matrix $q$:
\begin{itemize}
 \item $i$-th column of $d$ is odd if and only if coefficients $q_{ij}$ are even for $k+1\leq j \leq n$;
 \item $i$-th column of $d$ is odd if and only if one of the coefficients $q'_{ji}$ of the inverse matrix $q^{-1}$ is odd for some $1\leq j\leq k$.
\end{itemize}

In \cite{MW} M.~Westaway even conjectured that the above two properties are simultaneously fulfilled iff $p_d(i)=odd$, and thus, that
 it is possible to determine the type of summands appearing in $\K(A, F)$ from the given matrix $q$
 (Westaway's matrices $\nu$, $\mu$ appearing in the statement of \cite[Conjecture~9.1]{MW} correspond to our $q$ via $\nu = q^T$, $\mu = (q^{-1})^T$).
Although, as said before, this characterization works for the matrix $q$ obtained via the algorithm of~\cref{lm:normal-form},
 it may not work for {\it arbitrary} $q$ occurring in SNF decomposition of $\widetilde{A}$
 (e.\,g. the one obtained via the classical algorithm for finding SNF, which ignores columns' parities).
Thus, as stated, \cite[Conjecture~9.1]{MW} is false.

\begin{example}
To produce a counterexample, start with GCM 
 $A = \left(\begin{smallmatrix}
 2& -1& -2\\
-3& 2& -4\\
-4& -2& 2\end{smallmatrix}\right)$
  and write down SNF decomposition of~\cref{lm:normal-form} for $\widetilde{A}$:
\[\widetilde{A} = \left(\begin{smallmatrix}
-3& 1& 0 \\
-4& 0& 2 \\
 0& -2& 4
\end{smallmatrix} \right) = \left(\begin{smallmatrix}
 1& 0& 0\\
 0& 1& 0\\
-2& 2& 1\end{smallmatrix}\right) \cdot \mathrm{diag}(1,2,2) \cdot \left(\begin{smallmatrix}
-3& 1& 0\\
-2& 0& 1\\
 1& 0& 0 \end{smallmatrix}\right) = p \cdot d \cdot q.\]
Looking at the last column of $q$ we get that $1$st and $3$rd columns of $d$ are odd while $2$nd one is even,
 hence $\K(A, F) \cong \Kmw(F)/I \oplus \K(F)/2\K(F)$.

Now set $s:= t_{32}(1)$, $t:= t_{32}(-1) \in \SL(3, \ZZ)$ and consider the decomposition 
$\widetilde{A} = (p t) \cdot (t^{-1} d s^{-1}) \cdot (s q)$ which is still a valid SNF decomposition.
Now the application of Westaway's criterion to the matrix $sq$ incorrectly gives that the parity of the third column of $t^{-1}ds^{-1}=d$ is even rather than odd
 and, thus, gives another $\Kmw(F)/I$ summand instead of $\K(F)/2\K(F)$.
\end{example}

\printbibliography

\end{document}
